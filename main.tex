\documentclass{article}
\usepackage[utf8]{inputenc}
\usepackage{amssymb}

\title{MaxwellHelmholtz}
\author{Antoine Boissinot}
\date{March 2019}

\begin{document}

Prouver que l'équation d'onde de Maxwell
    $$\nabla^{2}\textbf{E}+  \mu_{0} \varepsilon \frac{\partial^2\textbf{E}}{\partial t^2}  =0$$
Peut s'écrire comme ceci     
    $$\nabla^{2}\textbf{A}+ k^{2}\textbf{A}=0$$
En utilisant la méthode de séparation de variables
     $$\textbf{E}=\textbf{A}(\textbf{r})\textbf{T}(\textbf{t})$$
    $$\nabla^{2}\textbf{A T}+  \mu_{0} \varepsilon \frac{\partial^2\textbf{A T}}{\partial t^2}  =0$$
    $$\frac{1}{\mu_{0} \varepsilon}  \frac{\nabla^{2}\textbf{A}}{\textbf{A}}  =- \frac{T''}{T} $$
\\
\\
Comme $$\textbf{r}=(x,y,z)$$
$$\nabla^{2}\textbf{\textbf{A}(\textbf{r})} = \frac{\partial^2\textbf{A}}{\partial r^2} = \textbf{A}''$$

On pose

    $$\frac{1}{\mu_{0} \varepsilon}  \frac{A''}{A}  =- \frac{T''}{T}=- \omega^2 $$

Comme 
$$k \equiv  \frac{ \omega n}{c} $$
$$c= \frac{1}{ \sqrt{\mu_{0} \varepsilon_{0}} } $$
$$n \equiv  \sqrt{ \frac{ \varepsilon }{\varepsilon_{0}} }  \equiv  \sqrt{\varepsilon_{r}} $$
il est possible de définir $k^2$ comme étant : 
$$k^2=\omega^2  \mu_{0} \varepsilon $$
Donc en prenant la partie spatiale de l'équation :
$$\frac{1}{\mu_{0} \varepsilon}  \frac{A''}{A} =- \omega^2 $$
$$\textbf{A''}+\omega^2\mu_{0} \varepsilon \textbf{A}=0$$
$$\textbf{A''}+k^2\textbf{A}=0 $$
$$\therefore \nabla^{2}\textbf{A}+ k^{2}\textbf{A}=0$$
\end{document}
